\documentclass[a4paper,11pt]{article}
\usepackage[margin=2cm]{geometry}
\usepackage[utf8]{inputenc}
\usepackage[french]{babel} % to write in french
\usepackage[final]{pdfpages} % include pdf files
\usepackage{enumitem}
\usepackage{eurosym}


\begin{document}

\begin{titlepage}
\begin{center}
	{\sc Université libre de Bruxelles}\\
	Faculté des Sciences\\
	Département d'Informatique
	\vfill{}\vfill{}

	{\Huge \par}{\Huge Decision Engineering}{\Huge \par}
	{\Huge \par}{\huge Cloud service providers comparison}{\Huge \par}

	{\huge \par}{\Large Victor Carakehian, }
				{\Large Xavier Barthel}{\huge \par}
	
	\vfill{}
	\includegraphics{img/ulb_symbol.pdf}

	\vfill{}\vfill{}
	Année académique 2014~--~2015
\end{center}
\end{titlepage}

\section{Introduction}
This project is about choosing an alternative amongst multiple other ones. The process of choosing an alternative is more or less simple when it is done on the basis of a single criterion. However, most of the time a decision involves more than one criterion for decision-making process and the complex decision problems can not be resolved on the basis of unidimensional approaches. The difficulty resides in evaluating alternatives taking in account criteria. Those criteria are often conflicting and do not use the same units to quantify them. The purpose of MCDA is to help decision makers by modeling and representing his preferences.\\ % TODO: MCDA provides a set of criteria aggregation methodologies.


\section{Context}
This project is about trying to find the best fitting cloud solution for the Railnova\footnote{https://www.railnova.eu/} company. Since the number of cloud providers and their services is always growing, it becomes difficult to make a wise choice without a good study. The situation is more complex than it appears, because the market is full of meaningless (catchy) keywords and the companies are doing there best to complicate some comparisons.


\section{Alternatives}
There is a minimum requirements needed by the user to perform the given tasks. This minimum requirements is a threshold to consider a service as a solution.\\

As a provider of telematics solutions for the railway world, Railnova has multiple needs of computation, data storage and connectivity capabilities. Mainly, the different types of cloud computing services can be referred to Software as a Service (SaaS), Infrastructure as a Service (IaaS) or Platform as a Service (PaaS). IaaS is an abstraction of the hardware for customers which need to work with virtual machines that they configure themselves. The PaaS type can be seen as an upper layer over IaaS solutions, consisting in a set of tools and libraries designed to make coding and deploying applications easier.\\

A set of offers from multiple companies has been chosen by consultation with Railnova's employees. 

\section{Criteria}
Several criteria are proposed in the literature for the evaluation of cloud services. The criteria which are used in this work has been chosen after consulting the company's preferences.
Those criteria are defined as:
\begin{description}[parsep=1pt,listparindent=\parindent,labelindent=\parindent,font=$\bullet$\ ]
  \item[Availability:] The average amount of time during which the service is accessible and operational.\\
    \emph{Note:} A very common way to evaluate the availability is to use the downtime per year.\\
    \emph{Measurement:} This criteria is expressed as a percentage of the downtime. The used formula is $\frac{t_{st}-t_{dt}}{t_{st}}*100$, where $t_{st}$ is the total service time and $t_{dt}$ is the total down-time.

  \item[Service efficiency:] How well does a service respond to a given task.\\
    \emph{Note:} The response time of the provider to accomplish a requested task emitted by the costumer.\\
    \emph{Measurement:} 

  \item[Cost:] The fee that the costumer must pay by using the given service.\\
    \emph{Note:} There is several type of pricing scheme and even a same provider offers different solutions which are suitable for the consumer requirements. For the same price, services can offer different amount of data storage, RAM, number of CPU cores, CPU frequency and bandwidth capacity. In addition, we must take in consideration the cost of what part of the offer price is dedicated to the human management price. Indeed, a solution costing 100\euro{} with a 10\% part dedicated to human cost is more likely to please Railnova than a solution costing 100\euro{} with a 80\% part dedicated to human cost. Because it supposes that the rest of the price is dedicated to the actual hardware and technology provided. This cost will be computed separately.\\
    \emph{Measurement:} The cost is defined with the performance metric made available by the provider by VM, $\frac{p}{cpu^a + net^b + data^c + RAM^d}$. Weights are used to express the importance of some features for the given application. Those weights must respect the following constraint $a+b+c+d=1$.

  \item[Reputation:] The reputation is the opinion of the user toward the provider.\\
    \emph{Note:} The reputation of a service provider has an impact on the productivity. An employee working with a solution in which he does not have motivation to work with may give rise to additional costs.\\
    \emph{Measurement:} This is expressed as a scale from 0 to 5.
    
  \item[Portability:] The ability to easily port a solution to another service provider.\\%(lock-in)
    \emph{Note:} \\
    \emph{Measurement:} 

  \item[Migration:] The effort needed to migrate from the current service to the new service.\\
    \emph{Note:} \\
    \emph{Measurement:} 

  \item[Usability:] The extent to which the service is easy to use and comprehend.\\
    \emph{Note:} Good documentation is useful when trying new features or in case of failure.\\
    \emph{Measurement:} A scale from 0 to 5 is used to express the subjective perception.

  \item[Reliability:] The reliability is the extent to which the provider meets the performance level agreed in the SLA.\\
    \emph{Note:} Availability failure compensation.\\
    \emph{Measurement:} 

  \item[Security backup:] the extent to which the client is confident in the data preservation capacity of the provider.\\
    \emph{Note:} \\
    \emph{Measurement:} 

  \item[Scalability:] Ability and cost to change the size of the resources used.\\
    \emph{Note:} \\
    \emph{Measurement:} 

\end{description}


\subsection{Weights}


\section{Analysis}


\end{document}
